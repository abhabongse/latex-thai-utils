%% thaienum.tex
%% Copyright 2017 Abhabongse Janthong
%
% This work may be distributed and/or modified under the
% conditions of the LaTeX Project Public License, either version 1.3
% of this license or (at your option) any later version.
% The latest version of this license is in
%   http://www.latex-project.org/lppl.txt
% and version 1.3 or later is part of all distributions of LaTeX
% version 2005/12/01 or later.
%
% This work has the LPPL maintenance status `maintained'.
% The Current Maintainer of this work is Abhabongse Janthong.
% This work consists of all files in the repository, including
% thaienum.sty and thaienum.tex.
%
% This file provides the basic howto usages of the package 'thaienum'.
%
\documentclass[10pt]{ltxguide}

\usepackage[thai,thaiindentfirst]{babel}
\usepackage[utf8x]{inputenc}
\usepackage[sans,sflaksaman,rmnorasi]{fonts-tlwg}
\usepackage{inconsolata}
\usepackage{enumitem}
\usepackage{thaienum}
\usepackage{multicol}
\usepackage{setspace}

% There is a bug when loading 'titlesec' so here is the quick hack. https://tex.stackexchange.com/questions/299969/titlesec-loss-of-section-numbering-with-the-new-update-2016-03-15
\usepackage[sf,bf]{titlesec}
\usepackage{etoolbox}
\makeatletter
\patchcmd{\ttlh@hang}{\parindent\z@}{\parindent\z@\leavevmode}{}{}
\patchcmd{\ttlh@hang}{\noindent}{}{}{}
\makeatother

\usepackage[usenames,dvipsnames]{color}
\usepackage[dvipsnames]{xcolor}
\definecolor{VeryLightGrayBackground}{gray}{0.96875}
\definecolor{LightGrayBackground}{gray}{0.9375}
\definecolor{DarkGray}{gray}{0.25}
\definecolor{Gray}{gray}{0.5}
\definecolor{LightGray}{gray}{0.75}
\definecolor{ShadeColor}{gray}{0.9}
\definecolor{BorderColor}{gray}{0.8}
\definecolor{ProcessBlue}{cmyk}{0.96,0,0,0}
\usepackage{listings}
\lstset{
    % Basic Settings
    basicstyle=\ttfamily,
    captionpos=b,
    % Margins and Background
    xleftmargin=3em,
    xrightmargin=1em,
    rulesep=1em,
    framextopmargin=5em,
    framexbottommargin=5em,
    framexleftmargin=1em,
    framexrightmargin=1em,
    backgroundcolor=\color{LightGrayBackground},
    % Spacing and Horizontal Flow
    showspaces=false,
    showstringspaces=false,
    showtabs=false,
    tabsize=2,
    breaklines=true,
    columns=fullflexible,
    keepspaces=true,
    % Vertical Flow
    aboveskip=\bigskipamount,
    belowskip=\bigskipamount,
    lineskip=0.00001pt,
    % Line Numbers
    stepnumber=1,
    numbers=left,
    numbersep=2em,
    numberstyle=\ttfamily\color{Gray},
    % Special Keyword Format
    stringstyle=\color{Maroon},
    keywordstyle=\color{NavyBlue},
    keywordstyle={[2]\color{teal}},
    commentstyle=\color{ForestGreen},
    extendedchars=true
}
\lstdefinelanguage{LaTeX}{
    language=[LaTeX]{TeX},
    sensitive=true
}
\usepackage{mdframed}

\setstretch{1.5}
\addtolength{\topmargin}{-3pc}
\addtolength{\textwidth}{6pc}
\addtolength{\oddsidemargin}{-2pc}
\addtolength{\textheight}{7pc}
\setlength{\parindent}{2pc}
\setlength{\parskip}{0pc}

\title{แพ็กเกจ \textlatin{\texttt{thaienum}}}
\author{อาภาพงศ์ จันทร์ทอง}

\date{เวอร์ชัน 0.1\\23 เมษายน 2017}

\begin{document}
\maketitle

\section{กล่าวนำ}

แพ็กเกจ \textlatin{\texttt{thaienum}} สำหรับ \textrm{\LaTeX} มีไว้สำหรับเขียนลำดับรายการผ่าน \emph{environment} ชื่อ \textlatin{\texttt{enumerate}} โดยสามารถเลือกใช้การนับแบบไทยได้หลายแบบ เช่น ก ข ค หรือ ๑ ๒ ๓ เป็นต้น

เดิมแล้วหากผู้ใช้งาน \textrm{\LaTeX} ต้องการที่จะพิมพ์ ๑ ๒ ๓ ในรายการ ผู้ใช้งานอาจจำเป็นต้องกำหนดเอง เพราะไม่สามารถกำหนดให้ \textrm{\LaTeX} ไล่รายการโดยอัตโนมัติได้  แพ็กเกจนี้จึงถูกสร้างขึ้นเพื่อแก้ปัญหาดังกล่าว

\section{วิธีใช้งานเบื้องต้น}

\subsection{ก่อนเริ่มใช้งาน}

ก่อนที่ผู้ใช้งานจะนำเข้าแพ็กเกจ \textlatin{\texttt{thaienum}} นีเพื่อใช้งาน ผู้นั้นจำเป็นต้องเรียกนำเข้าแพ็กเกจทั้งสิ้น 2 แพ็กเกจก่อนดังต่อไปนี้
\begin{enumerate}[topsep=0pc,itemsep=-0.25pc,label={\thaialph*.}]
    \item  แพ็กเกจ \textlatin{\texttt{babel}} และต้องนำเข้าการใช้งานภาษาไทยด้วย
    \item  แพ็กเกจ \textlatin{\texttt{enumitem}}
\end{enumerate}

\bigskip
โดยสรุปคือผู้ใช้งานสามารถใช้คำสั่งต่อไปนี้เพื่อเริ่มต้นใช้งานแพ็กเกจ \textlatin{\texttt{thaienum}} ได้

\begin{latintext}
\begin{lstlisting}[language=LaTeX,numbers=none]
\usepackage[thai]{babel}
\usepackage{enumitem}
\usepackage{thaienum}
\end{lstlisting}
\end{latintext}

\subsection{เมื่อเลือกใช้งาน}

เมื่อผู้ใช้งานต้องการจะเขียนรายการ สามารถสร้าง \emph{environment} ประเภท \textlatin{\texttt{enumerate}} โดยกำหนดค่าให้กับ \emph{parameter} ที่มีชื่อว่า \textlatin{\texttt{label}} ให้ตรงกับวิธีการนับที่ต้องการ (โปรดดูคู่มือของแพ็กเกจ \textlatin{\texttt{enumitem}} ประกอบ) สำหรับการนับแบบไทย ให้ใช้ค่าจากหนึ่งในรูปแบบดังต่อไปนี้

\begin{enumerate}[topsep=0pc,itemsep=-0.25pc,label={\thainum*.}]
    \item  \textlatin{\texttt{thaialph}} สำหรับการนับโดยใช้พยัญชนะไทย ก ข ค ง จ \ldots
    \item  \textlatin{\texttt{thaiAlph}} สำหรับการนับโดยใช้พยัญชนะไทย ก ข ฃ ค ฅ \ldots\, โดยไม่ข้าม ฃ ฅ และ ฆ
    \item  \textlatin{\texttt{thainum}} สำหรับการนับโดยใช้เลขไทย ๑ ๒ ๓ ๔ ๕ \ldots
    \item  \textlatin{\texttt{thaibracenum}} สำหรับการนับโดยใช้เลขไทย (๑) (๒) (๓) (๔) (๕) \ldots\, แบบมีวงเล็บ
\end{enumerate}

\bigskip
ในบางกรณีที่รายการมีความยาวมาก อาจทำให้พยัญชนะไทยไม่เพียงพอ สามารถเลือกใช้การนับแบบไทยเพิ่มเติมได้ดังนี้

\begin{enumerate}[topsep=0pc,itemsep=-0.25pc,label={\thainum*.},start=5]
    \item  \textlatin{\texttt{thaimultialph}} ซึ่งคล้ายกับ \textlatin{\texttt{thaialph}} แต่ว่าถัดจาก ฮ.นกฮูก นั้นรายการถัดไปจะนับใหม่เป็น กก กข กค กง กจ \ldots\, ไปเรื่อย ๆ
    \item  \textlatin{\texttt{thaimultiAlph}} ซึ่งคล้ายกับ \textlatin{\texttt{thaiAlph}} แต่ว่าถัดจาก ฮ.นกฮูก นั้นรายการถัดไปจะนับใหม่เป็น กก กข กฃ กค กฅ \ldots\, ไปเรื่อย ๆ
\end{enumerate}

\subsubsection*{ตัวอย่างการใช้งานที่ 1}
สมมติว่ารายการดังกล่าวเขียนในโค้ด \textrm{\LaTeX} ได้ว่า
\begin{latintext}
\begin{lstlisting}[language=LaTeX,numbers=none,escapechar={\^},basicstyle=\linespread{1.5}\ttfamily,lineskip={6pt}]
\begin{enumerate}[label={\thaialph*.}]
    \item  ^{\textthai{\small รายการที่หนึ่ง}}^
    \item  ^{\textthai{\small รายการที่สอง}}^
    \item  ^{\textthai{\small รายการที่สาม}}^
    \item  ^{\textthai{\small รายการที่สี่}}^
    \item  ^{\textthai{\small รายการที่ห้า}}^
    \item  ^{\textthai{\small รายการที่หก}}^
    \item  ^{\textthai{\small รายการที่เจ็ด}}^
    \item  ^{\textthai{\small รายการที่แปด}}^
\end{enumerate}
\end{lstlisting}
\end{latintext}

โค้ดดังกล่าวจะสร้างรายการดังนี้
\begin{quote}
    \color{DarkGray}
    \begin{enumerate}[topsep=0.25pc,itemsep=0pc,label={\thaialph*.}]
        \item  รายการที่หนึ่ง
        \item  รายการที่สอง
        \item  รายการที่สาม
        \item  รายการที่สี่
        \item  รายการที่ห้า
        \item  รายการที่หก
        \item  รายการที่เจ็ด
        \item  รายการที่แปด
    \end{enumerate}
\end{quote}

\subsubsection*{ตัวอย่างการใช้งานที่ 2}
สมมติว่ารายการดังกล่าวเขียนในโค้ด \textrm{\LaTeX} ได้ว่า
\begin{latintext}
\begin{lstlisting}[language=LaTeX,numbers=none,escapechar={\^},basicstyle=\linespread{1.5}\ttfamily,lineskip={6pt}]
\begin{enumerate}[label={\thainum*)}]
    \item  ^{\textthai{\small รายการที่หนึ่ง}}^
    \item  ^{\textthai{\small รายการที่สอง}}^
    \item  ^{\textthai{\small รายการที่สาม}}^
    \item  ^{\textthai{\small รายการที่สี่}}^
    \item  ^{\textthai{\small รายการที่ห้า}}^
    \item  ^{\textthai{\small รายการที่หก}}^
    \item  ^{\textthai{\small รายการที่เจ็ด}}^
    \item  ^{\textthai{\small รายการที่แปด}}^
\end{enumerate}
\end{lstlisting}
\end{latintext}

โค้ดดังกล่าวจะสร้างรายการดังนี้
\begin{quote}
    \color{DarkGray}
    \begin{enumerate}[topsep=0.25pc,itemsep=0pc,label={\thainum*)}]
        \item  รายการที่หนึ่ง
        \item  รายการที่สอง
        \item  รายการที่สาม
        \item  รายการที่สี่
        \item  รายการที่ห้า
        \item  รายการที่หก
        \item  รายการที่เจ็ด
        \item  รายการที่แปด
    \end{enumerate}
\end{quote}

\subsubsection*{ตัวอย่างการใช้งานที่ 3}
โปรดดู \emph{source code} ของไฟล์นี้ประกอบ ตัวอย่างนี้ต้องการแสดงให้เห็นว่าเมื่อรายการมีความยาวมากกว่าจำนวนพยัญชนะแล้ว การใช้งาน \textlatin{\texttt{thaimultialph}} จะทำให้รายการมีหน้าตาออกมาอย่างไร

\begin{multicols}{4}
    \small
    \begin{enumerate}[topsep=0em,itemsep=0em,label={\thaimultialph*.}]
        \item  First item.
        \item  Second item.
        \item  Third item.
        \item  Fourth item.
        \item  Other item.
        \item  Other item.
        \item  Other item.
        \item  Other item.
        \item  Other item.
        \item  Other item.
        \item  Other item.
        \item  Other item.
        \item  Other item.
        \item  Other item.
        \item  Other item.
        \item  Other item.
        \item  Other item.
        \item  Other item.
        \item  Other item.
        \item  Other item.
        \item  Other item.
        \item  Other item.
        \item  Other item.
        \item  Other item.
        \item  Other item.
        \item  Other item.
        \item  Other item.
        \item  Other item.
        \item  Other item.
        \item  Other item.
        \item  Other item.
        \item  Other item.
        \item  Other item.
        \item  Other item.
        \item  Other item.
        \item  Other item.
        \item  Other item.
        \item  Other item.
        \item  Other item.
        \item  Other item.
        \item  Other item.
        \item  Other item.
        \item  Other item.
        \item  Other item.
        \item  Other item.
        \item  Other item.
        \item  Other item.
        \item  Other item.
        \item  Other item.
        \item  Other item.
        \item  Other item.
    \end{enumerate}
    \begin{enumerate}[topsep=0em,itemsep=0em,label={({\thainum*})}]
        \item  First item.
        \item  Second item.
        \item  Third item.
        \item  Other item.
        \item  Other item.
        \item  Other item.
        \item  Other item.
        \item  Other item.
        \item  Other item.
        \item  Other item.
        \item  Other item.
        \item  Other item.
        \item  Other item.
        \item  Other item.
        \item  Other item.
        \item  Other item.
        \item  Other item.
        \item  Other item.
        \item  Other item.
        \item  Other item.
        \item  Other item.
    \end{enumerate}
\end{multicols}

\end{document}
