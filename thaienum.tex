%% thaienum.tex
%% Copyright 2017 Abhabongse Janthong
%
% This work may be distributed and/or modified under the
% conditions of the LaTeX Project Public License, either version 1.3c
% of this license or (at your option) any later version.
% The latest version of this license is in
%   https://www.latex-project.org/lppl/lppl-1-3c.txt
% and version 1.3c or later is part of all distributions of LaTeX
% version 2005/12/01 or later.
%
% This work has the LPPL maintenance status `maintained'.
% The Current Maintainer of this work is Abhabongse Janthong.
% This work consists of all files in the repository, including
% thaienum.sty and thaienum[.pre].tex.
%
% This file provides the basic howto usages of the package 'thaienum'.
%
\documentclass[11pt]{ltxguide}

%%------------------------------------------------------------------
%%  Load utility package
%%------------------------------------------------------------------
\usepackage{etoolbox}

%%------------------------------------------------------------------
%%  Set up Thai language and default font selection
%%------------------------------------------------------------------
\usepackage[T1]{fontenc}
\usepackage{libertine}
\usepackage[varqu,varl]{inconsolata}
\usepackage[thai,thaiindentfirst]{babel}
\usepackage[utf8x]{inputenc}
\usepackage[sans,sflaksaman,rmnorasi]{fonts-tlwg}

\renewcommand{\baselinestretch}{1.5}
\renewrobustcmd{\englishrmdefault}{LinuxLibertineT-TLF}
\renewrobustcmd{\englishsfdefault}{LinuxBiolinumT-TLF}
\renewrobustcmd{\englishttdefault}{zi4}

\renewrobustcmd{\thaittdefault}{zi4}
\renewrobustcmd{\ttfamily}{%
    \fontencoding{\latinencoding}%
    \fontfamily{\ttdefault}%
    \selectfont}

%%------------------------------------------------------------------
%% Load package to be documented
%%------------------------------------------------------------------
\usepackage{enumitem}
\usepackage{thaienum}
\setlist[enumerate,1]{%
    topsep=1pc,partopsep=0pc,parsep=0pc,itemsep=1pc,listparindent=2pc,%
    itemindent=0pc}

%%------------------------------------------------------------------
%%  Set default titlesec
%%  There is a bug when loading 'titlesec' so here is the quick hack. https://tex.stackexchange.com/questions/299969/titlesec-loss-of-section-numbering-with-the-new-update-2016-03-15
%%------------------------------------------------------------------
\usepackage[sf,bf]{titlesec}
\makeatletter
\patchcmd{\ttlh@hang}{\parindent\z@}{\parindent\z@\leavevmode}{}{}
\patchcmd{\ttlh@hang}{\noindent}{}{}{}
\makeatother
\usepackage{multicol}

%%------------------------------------------------------------------
%%  Setup color choices
%%------------------------------------------------------------------
\usepackage[dvipsnames,svgnames,table,fixpdftex,hyperref]{xcolor}
\definecolor{DarkGray}{gray}{0.25}
\definecolor{Gray}{gray}{0.5}
\definecolor{LightGray}{gray}{0.65}
\definecolor{VeryLightGray}{gray}{0.9}
\definecolor{GrayBG}{gray}{0.9}
\definecolor{LightGrayBG}{gray}{0.935}
\definecolor{VeryLightGrayBG}{gray}{0.97}

% Color settings for mdframed
\definecolor{ShadeColor}{gray}{0.935}
\definecolor{BorderColor}{gray}{0.785}

%%------------------------------------------------------------------
%%  Code listings package and settings
%%------------------------------------------------------------------
\RequirePackage{listings}
\lstset{%
    % Basic Settings
    basicstyle=\small\ttfamily,%
    captionpos=b,%
    % Margins and Background
    xleftmargin=0pc,%
    xrightmargin=0pc,%
    rulesep=2pc,%
    frame=trbl,%
    framesep=0.5pc,%
    framextopmargin=0pc,%
    framexbottommargin=0pc,%
    framexleftmargin=0.125pc,%
    framexrightmargin=0pc,%
    backgroundcolor=\color{ShadeColor},%
    rulecolor=\color{BorderColor},%
    % Spacing and Horizontal Flow
    showspaces=false,%
    showstringspaces=false,%
    showtabs=false,%
    tabsize=2,%
    breaklines=true,%
    columns=fullflexible,%
    keepspaces=true,%
    % Vertical Flow
    aboveskip=\bigskipamount,%
    belowskip=\bigskipamount,%
    lineskip=0.0001pt,
    % Line Numbers
    stepnumber=1,%
    numbers=left,%
    numbersep=1.25pc,%
    numberstyle=\ttfamily\color{Gray},%
    % Special Keyword Format
    stringstyle=\color{red},%
    commentstyle=\color{ForestGreen!90},%
    keywordstyle=\color{MidnightBlue!80},%
    keywordstyle={[2]\color{teal!80}},%
    extendedchars=true}

\lstdefinelanguage{LaTeX}{%
    language=[LaTeX]{TeX},%
    sensitive=true}

%%------------------------------------------------------------------
%%  Setup constant lengths for parts of document.
%%------------------------------------------------------------------
\addtolength{\topmargin}{-3pc}
\addtolength{\textwidth}{6pc}
\addtolength{\oddsidemargin}{-2pc}
\addtolength{\textheight}{7pc}
\setlength{\parindent}{2pc}
\setlength{\parskip}{0pc}

%%------------------------------------------------------------------
%%  Document information
%%------------------------------------------------------------------
\title{แพ็ก{\wbr}เก{\wbr}จ \textlatin{\texttt{thaienum}}}
\author{อาภา{\wbr}พงศ์ จันทร์{\wbr}ทอง}

\date{เวอร์ชัน 0.2\\30 เมษายน 2017}

\begin{document}
\maketitle

\section{กล่าว{\wbr}นำ}

โดย{\wbr}ปกติ{\wbr}แล้ว หาก{\wbr}ผู้{\wbr}ใช้งาน \textlatin{\textrm\LaTeX} ต้องการ{\wbr}ที่{\wbr}จะ{\wbr}เขียน{\wbr}ลำดับ{\wbr}รายการ{\wbr}โดย{\wbr}ที่{\wbr}สามารถ{\wbr}กำหนด{\wbr}ชนิด{\wbr}ของ{\wbr}หัว{\wbr}รายการ{\wbr}ใน{\wbr}รูปแบบ{\wbr}ต่าง ๆ มัก{\wbr}จะ{\wbr}เรียก{\wbr}ใช้งาน{\wbr}แพ็ก{\wbr}เก{\wbr}จ \texttt{enumitem} เพราะ{\wbr}เป็น{\wbr}แพ็ก{\wbr}เก{\wbr}จ{\wbr}ที่{\wbr}มี{\wbr}ความ{\wbr}ยืดหยุ่น{\wbr}สูง เช่น สามารถ{\wbr}กำหนด{\wbr}ให้{\wbr}ตั้ง{\wbr}หัว{\wbr}รายการ{\wbr}ด้วย{\wbr}เลข{\wbr}โรมัน{\wbr}ได้ เช่น{\wbr}

\smallskip
\renewcommand{\baselinestretch}{0.1}
\begin{lstlisting}[language=LaTeX,escapechar={â},basicstyle=\ttfamily,lineskip={0.05pc}]
\begin{enumerate}[label={\Roman*.}]
    \item  â{\textthai{\small รายการ{\wbr}ที่{\wbr}หนึ่ง}}â
    \item  â{\textthai{\small รายการ{\wbr}ที่{\wbr}สอง}}â
    \item  â{\textthai{\small รายการ{\wbr}ที่{\wbr}สาม}}â
    \item  â{\textthai{\small รายการ{\wbr}ที่{\wbr}สี่}}â
\end{enumerate}
\end{lstlisting}
\renewcommand{\baselinestretch}{1.5}

\begin{enumerate}[topsep=0pc,itemsep=0pc,label={\latintext\rmfamily\Roman*.}]
    \item รายการ{\wbr}ที่{\wbr}หนึ่ง{\wbr}
    \item รายการ{\wbr}ที่{\wbr}สอง{\wbr}
    \item รายการ{\wbr}ที่{\wbr}สาม{\wbr}
    \item รายการ{\wbr}ที่{\wbr}สี่{\wbr}
\end{enumerate}

\medskip
แต่{\wbr}ก็{\wbr}มี{\wbr}ผู้{\wbr}ใช้งาน \textlatin{\textrm\LaTeX} กลุ่ม{\wbr}หนึ่ง โดย{\wbr}เฉพาะ{\wbr}ผู้{\wbr}ใช้งาน{\wbr}ภาษา{\wbr}ไทย{\wbr}มัก{\wbr}ประสบ{\wbr}ปัญหา{\wbr}ที่{\wbr}ไม่{\wbr}สามารถ{\wbr}ตั้ง{\wbr}หัว{\wbr}รายการ{\wbr}เป็น{\wbr}เลข{\wbr}ไทย (เช่น ๑ ๒ ๓ ๔ \ldots) หรือ{\wbr}อักษร{\wbr}ภาษา{\wbr}ไทย (ก ข ค ง \ldots) ได้{\wbr}โดย{\wbr}อัตโนมัติ ก่อ{\wbr}ให้{\wbr}เกิด{\wbr}ความ{\wbr}ไม่{\wbr}สะดวก{\wbr}ใน{\wbr}การ{\wbr}ใช้งาน \textlatin{\textrm\LaTeX} กับ{\wbr}ภาษา{\wbr}ไทย{\wbr}เป็น{\wbr}อย่าง{\wbr}ยิ่ง{\wbr}

แพ็ก{\wbr}เก{\wbr}จ \textlatin{\texttt{thaienum}} สำหรับ \textlatin{\textrm\LaTeX} จึง{\wbr}ถูก{\wbr}สร้าง{\wbr}ขึ้น{\wbr}มา{\wbr}เพื่อ{\wbr}ตอบ{\wbr}โจทย์{\wbr}ดัง{\wbr}กล่าว โดย{\wbr}ใช้{\wbr}ควบคู่{\wbr}กับ{\wbr}แพ็ก{\wbr}เก{\wbr}จ{\wbr}หลัก{\wbr}อย่าง \texttt{enumitem} นั่นเอง{\wbr}


\section{วิธี{\wbr}ใช้งาน{\wbr}เบื้องต้น}

\subsection{ก่อน{\wbr}เริ่ม{\wbr}ใช้งาน}

ก่อน{\wbr}ที่{\wbr}ผู้{\wbr}ใช้งาน{\wbr}จะ{\wbr}นำ{\wbr}เข้า{\wbr}แพ็ก{\wbr}เก{\wbr}จ \texttt{thaienum} นีเพื่อ{\wbr}ใช้งาน ผู้{\wbr}นั้น{\wbr}จำเป็น{\wbr}ต้อง{\wbr}เรียก{\wbr}นำ{\wbr}เข้า{\wbr}แพ็ก{\wbr}เก{\wbr}จ{\wbr}ทั้งสิ้น 2 แพ็ก{\wbr}เก{\wbr}จ{\wbr}ก่อน{\wbr}ดัง{\wbr}ต่อ{\wbr}ไป{\wbr}นี้{\wbr}
\begin{enumerate}[topsep=0.25pc,itemsep=0pc,label={\thainum*.}]
    \item  แพ็ก{\wbr}เก{\wbr}จ \texttt{babel} และ{\wbr}ต้อง{\wbr}นำ{\wbr}เข้า{\wbr}การ{\wbr}ใช้งาน{\wbr}ภาษา{\wbr}ไทย{\wbr}ด้วย{\wbr}
    \item  แพ็ก{\wbr}เก{\wbr}จ \texttt{enumitem}
\end{enumerate}

\begin{lstlisting}[language=LaTeX]
\usepackage[thai]{babel}
\usepackage{enumitem}
\usepackage{thaienum}
\end{lstlisting}

\subsection{การ{\wbr}เลือก{\wbr}ใช้งาน}

เมื่อ{\wbr}ผู้{\wbr}ใช้งาน{\wbr}ต้องการ{\wbr}จะ{\wbr}เขียน{\wbr}รายการ{\wbr}ใหม่ สามารถ{\wbr}สร้าง \texttt{environment} ประะภท \texttt{enumerate} โดย{\wbr}กำหนด{\wbr}ค่า \texttt{parameter} ชื่อ \texttt{label} ให้{\wbr}มี{\wbr}หัว{\wbr}รายการ{\wbr}ตาม{\wbr}ที่{\wbr}ต้องการ{\wbr}ได้ โดย{\wbr}สามารถ{\wbr}กำหนด{\wbr}ค่า{\wbr}ต่อ{\wbr}ไป{\wbr}นี้{\wbr}

\begin{enumerate}[topsep=0.25pc,itemsep=0pc,label={\thainum*.}]
    \item  \texttt{thainum} สำหรับ{\wbr}การ{\wbr}นับ{\wbr}โดย{\wbr}ใช้{\wbr}เลข{\wbr}ไทย ๑ ๒ ๓ ๔ ๕ \ldots
    \item  \label{it:loalph} \texttt{thaialph} สำหรับ{\wbr}การ{\wbr}นับ{\wbr}โดย{\wbr}ใช้{\wbr}พยัญชนะ{\wbr}ไทย ก ข ค ง จ \ldots
    \item  \label{it:hialph} \texttt{thaiAlph} สำหรับ{\wbr}การ{\wbr}นับ{\wbr}โดย{\wbr}ใช้{\wbr}พยัญชนะ{\wbr}ไทย ก ข ฃ ค ฅ \ldots\, โดย{\wbr}ไม่{\wbr}ข้าม ฃ ฅ ฆ และ{\wbr}ยัง{\wbr}มี ฤ กับ ฦ เพิ่มเติม{\wbr}หลัง ร กับ ล ตาม{\wbr}ลำดับ{\wbr}อีก{\wbr}ด้วย{\wbr}
\end{enumerate}

\medskip
อย่างไร{\wbr}ก็{\wbr}ดี สำหรับ{\wbr}หัว{\wbr}รายการ{\wbr}ประเภท{\wbr}ที่ \ref{it:loalph} และ \ref{it:hialph} นั้น มี{\wbr}จำนวน{\wbr}พยัญชนะ{\wbr}จำกัด จึง{\wbr}ไม่{\wbr}สามารถ{\wbr}ใช้{\wbr}ใน{\wbr}กรณี{\wbr}ที่{\wbr}รายการ{\wbr}มี{\wbr}ความ{\wbr}ยาว{\wbr}เกิน 41 และ 46 รายการ{\wbr}ได้{\wbr}ตาม{\wbr}ลำดับ แพ็ก{\wbr}เก{\wbr}จ{\wbr}นี้{\wbr}จึง{\wbr}มี{\wbr}อนุญาต{\wbr}ให้{\wbr}กำหนด{\wbr}ค่า{\wbr}ได้{\wbr}อีก{\wbr}สอง{\wbr}ประเภท{\wbr}เพิ่มเติม ได้แก่{\wbr}

\begin{enumerate}[topsep=0.25pc,itemsep=0pc,start=4,label={\thainum*.}]
    \item  \texttt{thaimultialph} ซึ่ง{\wbr}คล้าย{\wbr}กับ \texttt{thaialph} แต่ว่า{\wbr}ถัด{\wbr}จาก ฮ.{\wbr}นก{\wbr}ฮูก นั้น{\wbr}รายการ{\wbr}ถัด{\wbr}ไป{\wbr}จะ{\wbr}นับ{\wbr}ใหม่{\wbr}เป็น กก กข กค กง กจ \ldots\, ไป{\wbr}เรื่อย ๆ
    \item  \texttt{thaimultiAlph} ซึ่ง{\wbr}คล้าย{\wbr}กับ \texttt{thaiAlph} แต่ว่า{\wbr}ถัด{\wbr}จาก ฮ.{\wbr}นก{\wbr}ฮูก นั้น{\wbr}รายการ{\wbr}ถัด{\wbr}ไป{\wbr}จะ{\wbr}นับ{\wbr}ใหม่{\wbr}เป็น กก กข กฃ กค กฅ \ldots\, ไป{\wbr}เรื่อย ๆ
\end{enumerate}

\newpage
\subsubsection{ตัวอย่าง{\wbr}การ{\wbr}ใช้งาน{\wbr}กับ{\wbr}เลข{\wbr}ไทย}

เรา{\wbr}สามารถ{\wbr}ใช้{\wbr}เลข{\wbr}ไทย{\wbr}กับ{\wbr}รายการ{\wbr}ได้{\wbr}ดังนี้{\wbr}

\smallskip
\renewcommand{\baselinestretch}{0.1}
\begin{lstlisting}[language=LaTeX,escapechar={â},basicstyle=\ttfamily,lineskip={0.05pc}]
\begin{enumerate}[label={\thainum*.}]
    \item  â{\textthai{\small รายการ{\wbr}ที่{\wbr}หนึ่ง}}â
    \item  â{\textthai{\small รายการ{\wbr}ที่{\wbr}สอง}}â
    \item  â{\textthai{\small รายการ{\wbr}ที่{\wbr}สาม}}â
    \item  â{\textthai{\small รายการ{\wbr}ที่{\wbr}สี่}}â
\end{enumerate}
\end{lstlisting}
\renewcommand{\baselinestretch}{1.5}

\begin{enumerate}[topsep=0pc,itemsep=0pc,label={\thainum*.}]
    \item  รายการ{\wbr}ที่{\wbr}หนึ่ง{\wbr}
    \item  รายการ{\wbr}ที่{\wbr}สอง{\wbr}
    \item  รายการ{\wbr}ที่{\wbr}สาม{\wbr}
    \item  รายการ{\wbr}ที่{\wbr}สี่{\wbr}
\end{enumerate}

\subsubsection{ตัวอย่าง{\wbr}การ{\wbr}ใช้งาน{\wbr}กับ{\wbr}อักษร{\wbr}ไทย}

เรา{\wbr}สามารถ{\wbr}ใช้{\wbr}อักษร{\wbr}ไทย{\wbr}กับ{\wbr}รายการ{\wbr}ได้{\wbr}ดังนี้{\wbr}

\smallskip
\renewcommand{\baselinestretch}{0.1}
\begin{lstlisting}[language=LaTeX,escapechar={â},basicstyle=\ttfamily,lineskip={0.05pc}]
\begin{enumerate}[label={\thaialph*.}]
    \item  â{\textthai{\small รายการ{\wbr}ที่{\wbr}หนึ่ง}}â
    \item  â{\textthai{\small รายการ{\wbr}ที่{\wbr}สอง}}â
    \item  â{\textthai{\small รายการ{\wbr}ที่{\wbr}สาม}}â
    \item  â{\textthai{\small รายการ{\wbr}ที่{\wbr}สี่}}â
\end{enumerate}
\end{lstlisting}
\renewcommand{\baselinestretch}{1.5}

\begin{enumerate}[topsep=0pc,itemsep=0pc,label={\thaialph*.}]
    \item  รายการ{\wbr}ที่{\wbr}หนึ่ง{\wbr}
    \item  รายการ{\wbr}ที่{\wbr}สอง{\wbr}
    \item  รายการ{\wbr}ที่{\wbr}สาม{\wbr}
    \item  รายการ{\wbr}ที่{\wbr}สี่{\wbr}
\end{enumerate}

\newpage
\subsubsection{ตัวอย่าง{\wbr}การ{\wbr}ใช้งาน{\wbr}กับ{\wbr}อักษร{\wbr}ไทย{\wbr}ใน{\wbr}รายการ{\wbr}ที่{\wbr}ยาว}

หาก{\wbr}เรา{\wbr}ตั้ง{\wbr}ค่า{\wbr}ให้ \lstinline[language=LaTeX]!label={\thaialph.}! กับ{\wbr}รายการ{\wbr}ที่{\wbr}ยาว{\wbr}เกิน 41 รายการ เรา{\wbr}จะ{\wbr}พบ{\wbr}ปัญหา{\wbr}ดังนี้ (โปรด{\wbr}ดู \texttt{source code} เพื่อ{\wbr}ความ{\wbr}ชัดเจน{\wbr}ที่{\wbr}มาก{\wbr}ขึ้น)
\begin{center}
    \color{red}\texttt{thaienum.tex:315: LaTeX Error: Counter too large. [ {\textbackslash}item ]}
\end{center}

\begin{multicols}{4}
    \scriptsize
    \begin{enumerate}[listparindent=0pc,topsep=0pc,itemsep=0pc,label={\thaialph*.}]
        \item  รายการ{\wbr}ที่{\wbr}หนึ่ง{\wbr}
        \item  รายการ{\wbr}ที่{\wbr}สอง{\wbr}
        \item  รายการ{\wbr}ที่{\wbr}สาม{\wbr}
        \item  รายการ{\wbr}ที่{\wbr}สี่{\wbr}
        \item  รายการ{\wbr}ที่{\wbr}ห้า{\wbr}
        \item  รายการ{\wbr}ที่{\wbr}หก{\wbr}
        \item  รายการ{\wbr}ที่{\wbr}เจ็ด{\wbr}
        \item  รายการ{\wbr}ที่{\wbr}แปด{\wbr}
        \item  รายการ{\wbr}ที่{\wbr}เก้า{\wbr}
        \item  รายการ{\wbr}ที่{\wbr}สิบ{\wbr}
        \item  รายการ{\wbr}ที่{\wbr}สิบ{\wbr}เอ็ด{\wbr}
        \item  รายการ{\wbr}ที่{\wbr}สิบ{\wbr}สอง{\wbr}
        \item  รายการ{\wbr}ที่{\wbr}สิบ{\wbr}สาม{\wbr}
        \item  รายการ{\wbr}ที่{\wbr}สิบ{\wbr}สี่{\wbr}
        \item  รายการ{\wbr}ที่{\wbr}สิบ{\wbr}ห้า{\wbr}
        \item  รายการ{\wbr}ที่{\wbr}สิบ{\wbr}หก{\wbr}
        \item  รายการ{\wbr}ที่{\wbr}สิบ{\wbr}เจ็ด{\wbr}
        \item  รายการ{\wbr}ที่{\wbr}สิบ{\wbr}แปด{\wbr}
        \item  รายการ{\wbr}ที่{\wbr}สิบ{\wbr}เก้า{\wbr}
        \item  รายการ{\wbr}ที่{\wbr}ยี่{\wbr}สิบ{\wbr}
        \item  รายการ{\wbr}ที่{\wbr}ยี่{\wbr}สิบ{\wbr}เอ็ด{\wbr}
        \item  รายการ{\wbr}ที่{\wbr}ยี่{\wbr}สิบ{\wbr}สอง{\wbr}
        \item  รายการ{\wbr}ที่{\wbr}ยี่{\wbr}สิบ{\wbr}สาม{\wbr}
        \item  รายการ{\wbr}ที่{\wbr}ยี่{\wbr}สิบ{\wbr}สี่{\wbr}
        \item  รายการ{\wbr}ที่{\wbr}ยี่{\wbr}สิบ{\wbr}ห้า{\wbr}
        \item  รายการ{\wbr}ที่{\wbr}ยี่{\wbr}สิบ{\wbr}หก{\wbr}
        \item  รายการ{\wbr}ที่{\wbr}ยี่{\wbr}สิบ{\wbr}เจ็ด{\wbr}
        \item  รายการ{\wbr}ที่{\wbr}ยี่{\wbr}สิบ{\wbr}แปด{\wbr}
        \item  รายการ{\wbr}ที่{\wbr}ยี่{\wbr}สิบ{\wbr}เก้า{\wbr}
        \item  รายการ{\wbr}ที่{\wbr}สาม{\wbr}สิบ{\wbr}
        \item  รายการ{\wbr}ที่{\wbr}สาม{\wbr}สิบ{\wbr}เอ็ด{\wbr}
        \item  รายการ{\wbr}ที่{\wbr}สาม{\wbr}สิบ{\wbr}สอง{\wbr}
        \item  รายการ{\wbr}ที่{\wbr}สาม{\wbr}สิบ{\wbr}สาม{\wbr}
        \item  รายการ{\wbr}ที่{\wbr}สาม{\wbr}สิบ{\wbr}สี่{\wbr}
        \item  รายการ{\wbr}ที่{\wbr}สาม{\wbr}สิบ{\wbr}ห้า{\wbr}
        \item  รายการ{\wbr}ที่{\wbr}สาม{\wbr}สิบ{\wbr}หก{\wbr}
        \item  รายการ{\wbr}ที่{\wbr}สาม{\wbr}สิบ{\wbr}เจ็ด{\wbr}
        \item  รายการ{\wbr}ที่{\wbr}สาม{\wbr}สิบ{\wbr}แปด{\wbr}
        \item  รายการ{\wbr}ที่{\wbr}สาม{\wbr}สิบ{\wbr}เก้า{\wbr}
        \item  รายการ{\wbr}ที่{\wbr}สี่{\wbr}สิบ{\wbr}
        \item  รายการ{\wbr}ที่{\wbr}สี่{\wbr}สิบ{\wbr}เอ็ด{\wbr}
        \item[.]  รายการ{\wbr}ที่{\wbr}สี่{\wbr}สิบ{\wbr}สอง{\wbr}
        % To replicate error, replace the line above with the line below.
        % \item  รายการ{\wbr}ที่{\wbr}สี่{\wbr}สิบ{\wbr}สอง{\wbr}
    \end{enumerate}
\end{multicols}

\medskip
แต่{\wbr}หาก{\wbr}เรา{\wbr}ตั้ง{\wbr}ค่า{\wbr}ให้ \lstinline[language=LaTeX]!label={\thaimultialph.}! กับ{\wbr}รายการ{\wbr}ที่{\wbr}ยาว{\wbr}เกิน 41 รายการ เรา{\wbr}จะ{\wbr}ได้{\wbr}ผลลัพธ์{\wbr}ดังนี้{\wbr}

\begin{multicols}{4}
    \scriptsize
    \begin{enumerate}[listparindent=0pc,topsep=0pc,itemsep=0pc,label={\thaimultialph*.}]
        \item  รายการ{\wbr}ที่{\wbr}หนึ่ง{\wbr}
        \item  รายการ{\wbr}ที่{\wbr}สอง{\wbr}
        \item  รายการ{\wbr}ที่{\wbr}สาม{\wbr}
        \item  รายการ{\wbr}ที่{\wbr}สี่{\wbr}
        \item  รายการ{\wbr}ที่{\wbr}ห้า{\wbr}
        \item  รายการ{\wbr}ที่{\wbr}หก{\wbr}
        \item  รายการ{\wbr}ที่{\wbr}เจ็ด{\wbr}
        \item  รายการ{\wbr}ที่{\wbr}แปด{\wbr}
        \item  รายการ{\wbr}ที่{\wbr}เก้า{\wbr}
        \item  รายการ{\wbr}ที่{\wbr}สิบ{\wbr}
        \item  รายการ{\wbr}ที่{\wbr}สิบ{\wbr}เอ็ด{\wbr}
        \item  รายการ{\wbr}ที่{\wbr}สิบ{\wbr}สอง{\wbr}
        \item  รายการ{\wbr}ที่{\wbr}สิบ{\wbr}สาม{\wbr}
        \item  รายการ{\wbr}ที่{\wbr}สิบ{\wbr}สี่{\wbr}
        \item  รายการ{\wbr}ที่{\wbr}สิบ{\wbr}ห้า{\wbr}
        \item  รายการ{\wbr}ที่{\wbr}สิบ{\wbr}หก{\wbr}
        \item  รายการ{\wbr}ที่{\wbr}สิบ{\wbr}เจ็ด{\wbr}
        \item  รายการ{\wbr}ที่{\wbr}สิบ{\wbr}แปด{\wbr}
        \item  รายการ{\wbr}ที่{\wbr}สิบ{\wbr}เก้า{\wbr}
        \item  รายการ{\wbr}ที่{\wbr}ยี่{\wbr}สิบ{\wbr}
        \item  รายการ{\wbr}ที่{\wbr}ยี่{\wbr}สิบ{\wbr}เอ็ด{\wbr}
        \item  รายการ{\wbr}ที่{\wbr}ยี่{\wbr}สิบ{\wbr}สอง{\wbr}
        \item  รายการ{\wbr}ที่{\wbr}ยี่{\wbr}สิบ{\wbr}สาม{\wbr}
        \item  รายการ{\wbr}ที่{\wbr}ยี่{\wbr}สิบ{\wbr}สี่{\wbr}
        \item  รายการ{\wbr}ที่{\wbr}ยี่{\wbr}สิบ{\wbr}ห้า{\wbr}
        \item  รายการ{\wbr}ที่{\wbr}ยี่{\wbr}สิบ{\wbr}หก{\wbr}
        \item  รายการ{\wbr}ที่{\wbr}ยี่{\wbr}สิบ{\wbr}เจ็ด{\wbr}
        \item  รายการ{\wbr}ที่{\wbr}ยี่{\wbr}สิบ{\wbr}แปด{\wbr}
        \item  รายการ{\wbr}ที่{\wbr}ยี่{\wbr}สิบ{\wbr}เก้า{\wbr}
        \item  รายการ{\wbr}ที่{\wbr}สาม{\wbr}สิบ{\wbr}
        \item  รายการ{\wbr}ที่{\wbr}สาม{\wbr}สิบ{\wbr}เอ็ด{\wbr}
        \item  รายการ{\wbr}ที่{\wbr}สาม{\wbr}สิบ{\wbr}สอง{\wbr}
        \item  รายการ{\wbr}ที่{\wbr}สาม{\wbr}สิบ{\wbr}สาม{\wbr}
        \item  รายการ{\wbr}ที่{\wbr}สาม{\wbr}สิบ{\wbr}สี่{\wbr}
        \item  รายการ{\wbr}ที่{\wbr}สาม{\wbr}สิบ{\wbr}ห้า{\wbr}
        \item  รายการ{\wbr}ที่{\wbr}สาม{\wbr}สิบ{\wbr}หก{\wbr}
        \item  รายการ{\wbr}ที่{\wbr}สาม{\wbr}สิบ{\wbr}เจ็ด{\wbr}
        \item  รายการ{\wbr}ที่{\wbr}สาม{\wbr}สิบ{\wbr}แปด{\wbr}
        \item  รายการ{\wbr}ที่{\wbr}สาม{\wbr}สิบ{\wbr}เก้า{\wbr}
        \item  รายการ{\wbr}ที่{\wbr}สี่{\wbr}สิบ{\wbr}
        \item  รายการ{\wbr}ที่{\wbr}สี่{\wbr}สิบ{\wbr}เอ็ด{\wbr}
        \item  รายการ{\wbr}ที่{\wbr}สี่{\wbr}สิบ{\wbr}สอง{\wbr}
        \item  รายการ{\wbr}ที่{\wbr}สี่{\wbr}สิบ{\wbr}สาม{\wbr}
        \item  รายการ{\wbr}ที่{\wbr}สี่{\wbr}สิบ{\wbr}สี่{\wbr}
        \item  รายการ{\wbr}ที่{\wbr}สี่{\wbr}สิบ{\wbr}ห้า{\wbr}
        \item  รายการ{\wbr}ที่{\wbr}สี่{\wbr}สิบ{\wbr}หก{\wbr}
        \item  รายการ{\wbr}ที่{\wbr}สี่{\wbr}สิบ{\wbr}เจ็ด{\wbr}
        \item  รายการ{\wbr}ที่{\wbr}สี่{\wbr}สิบ{\wbr}แปด{\wbr}
        \item  รายการ{\wbr}ที่{\wbr}สี่{\wbr}สิบ{\wbr}เก้า{\wbr}
        \item  รายการ{\wbr}ที่{\wbr}ห้า{\wbr}สิบ{\wbr}
    \end{enumerate}
\end{multicols}

\section{ขอ{\wbr}ขอบคุณ}

ขอ{\wbr}ขอบคุณ{\wbr}แพ็ก{\wbr}เก{\wbr}จ \texttt{moreenum} สำหรับ{\wbr}ความคิด{\wbr}ริเริ่ม{\wbr}และ{\wbr}แนวทาง{\wbr}ที่{\wbr}จะ{\wbr}สร้าง{\wbr}แพ็ก{\wbr}เก{\wbr}จ{\wbr}นี้{\wbr}ขึ้น{\wbr}มา และ{\wbr}ขอ{\wbr}ขอบคุณ{\wbr}แพ็ก{\wbr}เก{\wbr}จ \texttt{babel-thai} สำหรับ{\wbr}การ{\wbr}สนับสนุน{\wbr}ภาษา{\wbr}ไทย{\wbr}ใน \textlatin{\textrm\LaTeX} เรื่อย{\wbr}มา{\wbr}


\end{document}
